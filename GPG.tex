\title{GPG in Linux}
\author{Chari Karipidis\\
		2TinG\\}
\date{\today}

\documentclass[12pt]{article}

\usepackage{fancyhdr}
\pagestyle{fancy}
\lhead{}
\chead{}
\rhead{}
\lfoot{Chari Karipidis}
\cfoot{\thepage}
\rfoot{}
\renewcommand{\headrulewidth}{0.4pt}
\renewcommand{\footrulewidth}{0.4pt}

\begin{document}
\maketitle
\newpage
\tableofcontents
\newpage

\section{Introductie}
In dit document wordt het gebruik van GPG(GnuPG) nader verklaard.

\paragraph{Benadering}
De geschiedenis, werking en uitvoer wordt utgewerkt in Section~\ref{GPG} GPG.\\
Section~\ref{Belangrijke woorden} Belangrijke woorden, geeft een overzicht van belangrijke woorden in het document.

\section{GPG}\label{GPG}

\subsection{Geschiedenis}
Er is altijd wel een probleem met boodschappen verzenden en ontvangen, zonder dat men deze kunnen onderscheppen en lezen. Hier zijn handige uitvingen voor ontworpen, die helpen bij dit probleem.

\paragraph{Scytale}
In de tijd van de romeinen had men een manier nodig om berichten te versturen naar geallieerde troepen. Verzender en ontvanger waren in het bezit van een 'Scytale' van ieder dezelfde grootte. Dit voorwerp was een soort van cilinder. Hier werd een riem over gewikkeld en een boodschap op geschreven.\\
Bij het verwijderen van de riem, was deze tekst onleesbaar zonder behulp van de Scytale. De letters waren namelijk door mekaar. Bij ontvangts van de riem bij de troepen, wikkelde ze de riem over de Scytale die zij bezitte en was het zo mogelijk, de boodschap te lezen.\\
Dit was een soort van encryptie. Ervoor zorgen dat een onderschepper, de boodschap niet kan lezen.

\paragraph{Caesar methode}
Een andere encryptie-methode was de Caesar methode.\\
Deze bestond uit een zin hervormen m.b.v. het alphabet. Dit klinkt natuurlijk zeer logisch.
Het alphabet wordt namelijk gebruikt om zinnen te schrijven.\\
Maar na het schrijven van de nodige boodschap, wordt er een 'sleutel' gekozen. Deze sleutel is afgesproken cijfer tussen 1 en 26, tussen beide partijen.\\
Belangrijk is dat de cijfers overeenkomen met een letter uit het alphabet. Als het gekozen cijfer, 6 is. Wordt het alphabet 6 maal naar links verschoven. A wordt dan F en B wordt dan G, enz...\\
De ontvanger krijgt dan een wirwar van letters en kan deze ontcijferen door het alphabet terug te vormen foor het 6 maal naar rechts te verschuiven.

\subsection{Wat is GPG?}

*/ tabellen / figuren / lijst van tabellen / lijst van figuren*

\section{Belangrijke woorden}\label{Belangrijke woorden}
*weergave van moeilijke woorden in tabelvorm*

\bibliographystyle{abbrv}
\bibliography{myID}

\end{document}
