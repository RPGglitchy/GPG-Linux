\title{GPG in Linux}
\author{Chari Karipidis\\
		2TinG\\}
\date{\today}

\documentclass[12pt]{article}

\begin{document}
\maketitle
\newpage
\tableofcontents
\newpage

\section{Introductie}
In dit document wordt het gebruik van GPG(GnuPG) nader verklaard.


\paragraph{Benadering}
De geschiedenis, werking en uitvoer wordt utgewerkt in Section~\ref{GPG} GPG.\\
Section~\ref{Belangrijke woorden} Belangrijke woorden, geeft een overzicht van belangrijke woorden in het document.

\section{GPG}\label{GPG}
*tekst / tabellen / figuren / lijst van tabellen / lijst van figuren*

\section{Belangrijke woorden}\label{Belangrijke woorden}
*weergave van moeilijke woorden in tabelvorm*

\bibliographystyle{abbrv}
\bibliography{myID}

\end{document}
